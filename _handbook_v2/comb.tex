\section{Combination 组合数学}
\subsection{Combination Identities 组合恒等式}
${n}\choose{r}$ is the number of r-element subsets of an n-element set\cite{comb1}, and
\begin{equation}
{{n}\choose{r}} = \frac{n!}{r!(n-r)!}
\end{equation}
Mirror Identity 对称性质
\begin{equation}
{{n}\choose{r}} = {{n}\choose{n-r}}
\end{equation}

\subsubsection{Pascal's Triangle}
Pascal's Triangle Identity\cite{comb1}.
\begin{equation}
{{n}\choose{r}} = {{n-1}\choose{r-1}} + {{n-1}\choose{r}}
\end{equation}
Extended Pascal's Triangle Identity\cite{comb1}.
\begin{equation}
{{n}\choose{r}} = \sum \limits_{i=0}^k{{{n-k}\choose{r-k-i}}{{k}\choose{i}}}
\end{equation}

\subsubsection{Row's sum}
Binomial theorem. 二项式定理
\begin{equation}
\sum \limits_{k=0}^n{{n}\choose{k}} = 2^n
\end{equation}
平方和:The sum of the squares of the elements of row n equals the middle element of row $(2n - 1)$ \cite{comb2}.
\begin{equation}
\sum \limits_{k=0}^n{{{n}\choose{k}}^2} = {{2n}\choose{n}}
\end{equation}

\subsubsection{Diagonal's sum}
Diagonal sum\cite{comb1}. 对角和
\begin{equation}
\sum \limits_{i=r}^n{{i}\choose{r}} = {{n+1}\choose{r+1}}
\end{equation}
Second order diagonal sum\cite{comb1}.
\begin{equation}
\sum \limits_{i=r}^n{(n-i+1){{i}\choose{r}}} = {{n+2}\choose{r+2}}
\end{equation}
K-th order diagonal sum\cite{comb1}.
\begin{equation}
\sum \limits_{i=r}^n{{{n-i+k-1}\choose{k-1}}{{i}\choose{r}}} = {{n+k}\choose{r+k}}
\end{equation}
The ``shallow diagonals'' of Pascal's triangle sum to Fibonacci numbers\cite{comb3},
\begin{equation}
\sum \limits_{k=0}^{\lfloor n/2 \rfloor}{{n-k}\choose{k}} = F_{n+1}
\end{equation}

\subsection{Figurate number 形数}
The simplicial polytopic numbers for r = 1, 2, 3, 4, ... are:\cite{comb4}

\begin{equation}
P_1(n) = \frac{n}{1} = {{n+0}\choose{1}}
\end{equation}
\begin{equation}
P_2(n) = \frac{n(n+1)}{2} = {{n+1}\choose{2}} \mbox{(triangular numbers)}
\end{equation}
\begin{equation}
P_3(n) = \frac{n(n+1)(n+2)}{6} = {{n+2}\choose{3}}
\end{equation}
\begin{equation}
P_4(n) = \frac{n(n+1)(n+2)(n+3)}{24} = {{n+3}\choose{4}}
\end{equation}
\ldots
\begin{equation}
P_r(n) = \frac{n(n+1)(n+2)\ldots(n+r-1)}{r!} = {{n+r-1}\choose{r}}
\end{equation}

\subsection{Catalan number}
