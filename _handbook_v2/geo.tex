\newcommand{\cgeo}[1]{\lstinputlisting[language=C++]{code/computational-geometry/#1}}

\section{计算几何}

\subsection{二维几何}
\subsubsection{基本函数}
\cgeo{2D.d/00-2D-common.cc}
\subsubsection{判相交}
\cgeo{2D.d/01-2D-is-intersect.cc}
\subsubsection{点在线上投影(project)/对称点(reflect)}
\cgeo{2D.d/02-2D-proj-ref.cc}
\subsubsection{点点/点线/线段……各种距离}
\cgeo{2D.d/03-2D-distance.cc}
\subsubsection{直线交点}
\cgeo{2D.d/04-2D-cross-line.cc}
\subsubsection{各种三角形}
\cgeo{2D.d/05-2D-triangle.cc}
\subsubsection{多边形}
\cgeo{2D.d/06-2D-polygon.cc}
\subsubsection{圆}
\cgeo{2D.d/07-2D-circle.cc}

\subsection{三维几何}
\subsubsection{新版本}
\cgeo{3D.d/geo3d_v2.1.cpp}
% \subsubsection{新版本}
% \cgeo{3D.d/geo3d_v2.cpp}
% \subsubsection{旧用版本}
\cgeo{3D.d/geo3d.cpp}

\subsection{凸包}
\subsubsection{二维单调链}
\cgeo{convex-hull/Monotone_Chain_Convex.cpp}
\subsubsection{三维凸包}
\cgeo{convex-hull/ch3d-tmp.cpp}


\subsection{圆并/交}
\subsubsection{SPOJ/CIRU,VCIRCLES}
多圆面积并的面积
\cgeo{2D.d/undering/circle-union/SPOJ-CIRU-VCIRCLES.cc}
\subsubsection{SPOJ/CIRUT}
扩展圆并,求交了k次的面积
\cgeo{2D.d/undering/circle-union/SPOJ-CIRUT.cc}
\subsubsection{SGU/435}
求交了奇数次和偶数次的面积
\cgeo{2D.d/undering/circle-union/SGU-435.cc}
\subsubsection{CII/4492,HDU/3239}
求圆并减去圆交,trick:要去除相同的圆
然后求去除相同圆后k个圆的交
\cgeo{2D.d/undering/circle-union/CII-4492-HDU-3239.cc}
\subsubsection{CII/4530}
求最多几个圆相交于一块区域,以及区域的块数
\cgeo{2D.d/undering/circle-union/CII-4530.cc}

\subsection{圆与线}
\subsubsection{HDU/4116}
求平面一直线最多能交几个圆。

推论:最优直线可为某一圆切线。
\cgeo{2D.d/undering/HDU-4116-line-cut-circle.cc}

\subsection{圆凸包}
\subsubsection{CII/6012}
多圆求凸包,计算任意两圆不相交切线的四个切点,将所有点做凸包即可。
\cgeo{2D.d/undering/CII-6012-circle-convex.cc}

\subsection{圆环区间统计}
\subsubsection{POJ/4048}
由一起点发射的射线最多交多少线段
\cgeo{2D.d/undering/POJ-4048-line-cut-segment.cc}

\subsection{圆与多边形交}
\subsubsection{POJ/2986,3675,ZOJ/2675,HDU/4404}
\cgeo{2D.d/undering/circle-polygon-area.cpp}
